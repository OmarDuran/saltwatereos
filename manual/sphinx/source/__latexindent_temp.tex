@article{Driesner2007Part2,
abstract = {A set of correlations for the volumetric properties and enthalpies of phases in the system H2O-NaCl as a function of temperature, pressure, and composition has been developed that yields accurate values from 0 to 1000 °C, 1 to 5000 bar, and 0 to 1 XNaCl. The volumetric properties of all fluid phases from low-density vapor to hydrous salt melts and single-phase binary fluids at high pressures and temperatures, can be described by a simple equationVsolution (T, P, XNaCl) = VH2 O (n1 + n2 T, P)i.e., for a given pressure and composition, the molar volume of the solution is related to the molar volume of pure water at the same pressure by a linear scaling of temperature. The parameters n1 and n2 are simple functions of pressure and composition. This linear relation could be demonstrated for all (P, XNaCl) pairs where accurate volumetric data are available over a sufficiently large temperature interval. Extrapolations over as much as 300 °C predict high temperature data within their experimental uncertainty of 1-2{\%}. With a simple fit of parameters n1 and n2, the vast majority of several thousand experimental data points from the literature can be reproduced within experimental error, including dilute solutions in the compressible region. Although a strict theoretical foundation is lacking, this behavior can phenomenologically be rationalized as reflecting the relation between the linear to near-linear isochores of binary fluids to those of pure water. Accordingly, small deviations from linearity that amount to 0.1-0.2{\%} error in molar volume occur only at low temperatures where the pure water isochores behave non-linearly. This effect is accounted for by introducing a deviation function. Given its high accuracy, the formulation for the molar volumes could be integrated along isotherms-isobars to generate data for the specific enthalpy of binary solutions. In order to allow direct computation rather than numerical integration, a correlation scheme was developed that is mathematically similar to that for molar volumes. Specific enthalpies computed from the correlation typically agree within 1-3{\%} with those obtained from other studies. Similarly, isobaric heat capacities show good agreement with published data except for high salinities at moderate pressures and temperatures. {\textcopyright} 2007 Elsevier Ltd. All rights reserved.},
author = {Driesner, Thomas},
doi = {10.1016/j.gca.2007.05.026},
file = {:Users/zguo/Documents/Library.papers3/Files/8C/8CB5CB13-3460-481D-8FCD-B9E93F143493.pdf:pdf},
issn = {00167037},
journal = {Geochimica et Cosmochimica Acta},
mendeley-groups = {SaltWaterEOS},
number = {20},
pages = {4902--4919},
title = {{The system H2O-NaCl. Part II: Correlations for molar volume, enthalpy, and isobaric heat capacity from 0 to 1000 °C, 1 to 5000 bar, and 0 to 1 XNaCl}},
volume = {71},
year = {2007}
}
@article{Driesner2007Part1,
abstract = {Realistic simulations of fluid flow in geologic systems have severely been hampered by the lack of a consistent formulation for fluid properties for binary salt-water fluids over the temperature-pressure-composition ranges encountered in the Earth's crust. As the first of two companion studies, a set of correlations describing the phase stability relations in the system H2O-NaCl is developed. Pure water is described by the IAPS-84 equation of state. New correlations comprise the vapor pressure of halite and molten NaCl, the NaCl melting curve, the composition of halite-saturated liquid and vapor, the pressure of vapor + liquid + halite coexistence, the temperature-pressure and temperature-composition relations for the critical curve, and the compositions of liquid and vapor on the vapor + liquid coexistence surface. The correlations yield accurate values for temperatures from 0 to 1000 °C, pressures from 0 to 5000 bar, and compositions from 0 to 1 XNaCl (mole fraction of NaCl). To facilitate their use in fluid flow simulations, the correlations are entirely formulated as functions of temperature, pressure and composition. {\textcopyright} 2007 Elsevier Ltd. All rights reserved.},
author = {Driesner, Thomas and Heinrich, Christoph A.},
doi = {10.1016/j.gca.2006.01.033},
file = {:Users/zguo/Documents/Library.papers3/Files/4C/4C8053D8-86E7-4539-A07F-B338154B7A20.pdf:pdf},
issn = {00167037},
journal = {Geochimica et Cosmochimica Acta},
mendeley-groups = {SaltWaterEOS},
number = {20},
pages = {4880--4901},
title = {{The system H2O-NaCl. Part I: Correlation formulae for phase relations in temperature-pressure-composition space from 0 to 1000 °C, 0 to 5000 bar, and 0 to 1 XNaCl}},
volume = {71},
year = {2007}
}
