\section{Installation}
\label{sec:installation}
OpenFOAM is required before running \foam, 
see \href{https://openfoam.org}{homepage} of OpenFOAM for detailed instructions of installation on 
\href{https://openfoam.org/download/7-ubuntu/}{Ubuntu},
\href{https://openfoam.org/download/7-linux/}{Linux},
\href{https://openfoam.org/download/7-macos/}{Mac OS}
and 
\href{https://openfoam.org/download/windows/}{Windows}
system.

\subsection{Check OpenFOAM}
Before compiling \foam, it is better to do the following steps to ensure OpenFOAM is installed successfully.

%\subsubsection{Copy and recompile a solver}
Copy a solver, e.g., \mintinline{bash}{buoyantPimpleFoam}, from source code directory of OpenFOAM to another folder whatever you like, e.g., the download folder of \mintinline{bash}{~/Download} and compile it, if there is no error and you can get the executable file of the solver, congratulations! The OpenFOAM is installed correctly.

\begin{itemize}
	\item  Copy sorce code of solver to a new directory
	\begin{minted}[bgcolor=\bgcolor_Code,tabsize=2]{bash}
		cp -rf $FOAM_SOLVERS/heatTransfer/buoyantPimpleFoam ~/Downloads
	\end{minted}
	\item Change path of executable file of the solver
	\begin{minted}[bgcolor=\bgcolor_Code, tabsize=2]{bash}
	cd ~/Downloads/buoyantPimpleFoam
	ls Make/files
	\end{minted}
	open \mintinline{bash}{Make/files} using your favorite editor, e.g. Visual Studio Code, and then replace \textcolor{red}{FOAM\_APPBIN} by \textcolor{red}{FOAM\_USER\_APPBIN}
	
	\item Compile solver
	\begin{minted}[bgcolor=\bgcolor_Code, tabsize=2]{bash}
	wmake
	\end{minted}
	\item Check the executable file of the solver
	\begin{minted}[bgcolor=\bgcolor_Code, tabsize=2]{bash}
	ls $FOAM_USER_APPBIN
	\end{minted}
	The solver is compiled successfully if \mintinline{bash}{buoyantPimpleFoam} is in the directory of \mintinline{bash}{$FOAM_USER_APPBIN}.
\end{itemize}

\subsection{Get source code of \foam}
The source code of \foam can be downloaded from Github repository:
 \href{https://github.com/zguoch/HydrothermalFoam}{\nolinkurl{https://github.com/zguoch/HydrothermalFoam}}.
 Alternatively, \foam can also be cloned using \href{https://git-scm.com}{git} commond,
 \begin{minted}[tabsize=2,bgcolor=\bgcolor_Code]{bash}
 	git clone https://github.com/zguoch/HydrothermalFoam.git
 \end{minted}
\subsection{Compile \foam}
\begin{itemize}
	\item  Complie solvers
	\begin{minted}[tabsize=2,bgcolor=\bgcolor_Code]{bash}
		cd solvers
		wmake
	\end{minted}
	
	\item Compile boundary condtions
	
	\begin{minted}[tabsize=2,bgcolor=\bgcolor_Code]{bash}
		cd libraries/BoundaryConditions
		wmake
	\end{minted}
	
	\item Compile thermal physical models
	\begin{minted}[tabsize=2,bgcolor=\bgcolor_Code]{bash}
		cd libraries/ThermoModels
		wmake
	\end{minted}
	
	\item Compile freesteam \\
	the source code of \href{http://freesteam.sourceforge.net/compile.php}{freesteam} is configured by 
	\href{https://scons.org}{scons} which is a software construction tool just like \href{https://en.wikipedia.org/wiki/CMake}{cmake}, and scons is based on python2.
	Installation of scons is pretty easy, just using command of \mintinline{bash}{pip install scons} if you have installed anaconda2 on the host system.
	Then using the following command to compile freesteam library (\mintinline{bash}{libfreesteam.dylib}), which will be used by \foam.
	\begin{minted}[tabsize=2,bgcolor=\bgcolor_Code]{bash}
		cd libraries/freesteam-2.1
		scons INSTALL_PREFIX=/usr/local install
	\end{minted}
\end{itemize}

